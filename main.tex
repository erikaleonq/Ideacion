\documentclass{article}
\usepackage[utf8]{inputenc}
\usepackage[spanish]{babel}
\usepackage{listings}
\usepackage{graphicx}
\graphicspath{ {images/} }
\usepackage{cite}

\begin{document}

\begin{titlepage}
    \begin{center}
        \vspace*{1cm}
            
        \Huge
        \textbf{Proyecto final: Los primeros pasos}
            
        \vspace{0.5cm}
        \LARGE
        Ideación
            
        \vspace{1.5cm}
            
        \textbf{Juan Pablo Cruz Gómez\\
                Erika Dayana León Quiorga}
            
        \vfill
            
        \vspace{0.8cm}
            
        \Large
        Despartamento de Ingeniería Electrónica y Telecomunicaciones\\
        Universidad de Antioquia\\
        Medellín\\
        Septiembre de 2021
            
    \end{center}
\end{titlepage}

\tableofcontents
\newpage
<<<<<<< HEAD
\section{DESCRIPCIÓN DEL JUEGO} \label{contenido}
\begin{enumerate}
  \item \textbf{IDEA GENERAL:} Es un juego en el que el jugador tendrá que moverse por la pantalla evitando los diferentes obstáculos que se encuentran en ella. El objetivo será llegar a un a un lugar determinado en la pantalla en donde se pasará al siguiente nivel. Cada jugador iniciará con un número determinado de vidas para luego ir recolectando más en cada nivel, si las vidas se agotan el juego se reiniciará en el primer nivel.
  \item \textbf{AVATAR:} El personaje principal de nuestro juego irá avanzando a través del mapa sorteando los diferentes obstáculos y proyectiles presentados, para sortearlos tendrá la posibilidad de saltar los obstáculos y esquivar o contraatacar los proyectiles lanzando otros. En cuando a su aspecto físico trataremos de mantenerlo simple sin dejar atrás la estética del juego.
  \item \textbf{ENTORNO:} Nuestro objetivo como desarrolladores del juego será que este sea lo más funcional y fluido posible, es por esto que la estética visual se mantendrá simple pero atractiva para el jugador. A continuación se mencionarán algunas características del entorno:
    \begin{itemize}
        \item Plataformas: Será por donde el avatar irá avanzando para llegar a la meta, se presentan en varios niveles horizontales a los cuales el avatar accederá saltando a través de la pantalla. En cada una de estas se encontrarán obstáculos, proyectiles y herramientas.
        \item Obstáculos:
        \newline
        -	Plataforma rota: Este obstáculo será presentado como explosiones que romperán la plataforma por la cual se desplaza el avatar, las cuales irán explotando al momento de iniciar el juego, es decir, visualmente al iniciar se verán todas las plataformas completas, pero trascurrirán una serie de explosiones que darán lugar a un espacio en la plataforma que el avatar debe saltar.
        \newline
        -   Exámenes materia bajo rendimiento: Uno de los obstáculos que deberá ser evitado por el jugador será el de los exámenes, estarán repartidos en lugares determinados de la pantalla y si son colisionados por el avatar resultará en la pérdida de una de sus vidas.
        \item Proyectiles: Estos serán representaciones de hojas de quizes y serán lanzados por el antagonista a lo largo de cada una de las plataformas. Deberán ser esquivados por el avatar o ser contrarrestados con “Contraataques de conocimiento”.
        \item Herramientas: Estas serán representaciones de libros que se irán recogiendo a lo largo de las plataformas para ser usadas como contraataque a los proyectiles. También se repartirán por la pantalla algún tipo de recompensas que el jugador irá recolectando para sumar puntuación, la que ayudará a la obtención de más vidas.
    \end{itemize}
  \item \textbf{ANTAGONISTA:} El antagonista será una representación de un profesor que lanzará los proyectiles en algunas de las plataformas y se irá moviendo de forma vertical por la pantalla en el costado izquierdo de esta. El antagonista no tendrá variaciones en el transcurso del juego.
  \item \textbf{DIFICULTAD:} La dificultad aumentará a medida que avance el juego, esto se verá reflejado en la cantidad de obstácúlos en pantalla y la velocidad con la que los proyectiles son lanzados.
\end{enumerate}
=======
\section{Introducción}\label{intro}
En este informe se quiere plasmar la idea inicial para el proyecto final de la materia informática 2 pensada por los dos integrantes del grupo. Se proporcionará una vista general del mismo recopilar las opiniones de cada integrante.

\section{Descripción del juego.} \label{contenido}

  La idea general es un juego en el que haya que evitar obstáculos. los jugadores deben moverse por la pantalla tratando de no tocar los diferentes obstáculos que se encuentran en ella, el objetivo sería llegar a un lugar determinado de la pantalla en donde está el pase al siguiente nivel. cada jugador iniciaría con un número determinado de vidas para luego ir recolectando más en cada nivel, si las vidas se agotan el juego se reiniciará en el primer nivel.

Profundizando en los diferentes obstáculos y niveles del juego,
cada nivel corresponde a un salón de la universidad en el cual se pueden dictar varias materias por semestre, antes de iniciar el juego cada jugador debe de elegir una materia dictada en ese salón (nivel) en el cual crea le va muy mal.

En el mapa del nivel cada jugador se va encontrar con 5 obstáculos correspondientes a parciales de la materia que escogió los cuales no puede tocar o de lo contario perderían la partida e iniciarían desde cero, un jugador si puede pasar tocando los parciales de la materia que escogió el otro ya que no es la materia que le va muy mal y no lo mataría.

Además de los 5 obstáculos para cada uno, se tendrán otra serie de obstáculos que controlar a los 2 jugadores, uno de estos obstáculos consistiría en un avatar de un profesor disparando quizes como misiles, los cuales se deben esquivar o también contrarrestar disparando libros que se pueden ir recogiendo mientras se avanza por el mapa.

Si un jugador muere los dos deben repetir el nivel ya que la idea es que los 2 pases juntos todos los niveles para poder ganar el juego.

>>>>>>> 11f032bdde5ebb6950622330efe45d1d6dc56f98


\end{document}
