\ documentclass { artículo }
	\ usepackage [ utf8 ] { inputenc }
	\ usepackage [ español ] { babel }
	\ usepackage { listados }
	\ usepackage { graphicx }
	\ graphicspath {{imágenes /}}
	\ usepackage { cite }
	

	\ begin { document }
	

	\ begin { titlepage }
	    \ begin { center }
	        \ vspace * {1cm}
	            
	        \Enorme
	        \ textbf { Proyecto final: Los primeros pasos }
	            
	        \ vspace {0,5 cm}
	        \GRANDE
	        Ideación
	            
	        \ vspace {1,5 cm}
	            
	        \ textbf { Juan Pablo Cruz Gómez \\
	                Erika Dayana León Quiorga }
	            
	        \ vfill
	            
	        \ vspace {0,8 cm}
	            
	        \Grande
	        Despartamento de Ingeniería Electrónica y Telecomunicaciones \\
	        Universidad de Antioquia \\
	        Medellín \\
	        Marzo de 2021
	            
	    \ end { centro }
	\ end { página de título }
	

	\Tabla de contenido
	\nueva pagina
	\ section { Introducción } \ label { intro }
	En este informe se quiere plasmar la idea inicial para el proyecto final de la materia informática 2 pensada por los dos integrantes del grupo. Se dará una vista general del mismo recopilando las opiniones de cada integrante.
	

	\ section { Descripción del juego. } \ label { contenido }
	\ begin { enumerate }
	  \ item La idea general es un juego en el que haya que evitar obstáculos. los jugadores deben moverse por la pantalla tratando de no tocar los diferentes obstáculos que se encuentran en ella, el objetivo sería llegar a un lugar determinado de la pantalla en donde está el pase al siguiente nivel. cada jugador iniciaría con un número determinado de vidas para luego ir recolectando más en cada nivel, si las vidas se agotan el juego se reiniciará en el primer nivel.

Profundizando en los diferentes obstáculos y niveles del juego,
cada nivel correspondería a un salón de la universidad en el cual se pueden dictar varias materias por semestre, antes de iniciar el juego cada jugador debe de elegir una materia dictada en ese salón (nivel) en la cual crea le va muy mal.

En el mapa del nivel cada jugador se va encontrar con 5 obstáculos correspondientes a parciales de la materia que escogió los cuales no puede tocar o de lo contario perderían la partida e iniciarían desde cero, un jugador si puede pasar tocando los parciales de la materia que escogió el otro ya que no es la materia que le va muy mal y no lo mataría.

Además de los 5 obstáculos para cada uno, se tendrán otra serie de obstáculos que afectan a los 2 jugadores, uno de estos obstáculos consistiría en un avatar de un profesor disparando quizes como misiles, los cuales se deben esquivar o también contrarrestar disparando libros que se pueden ir recogiendo mientras se avanza por el mapa.

Si un jugador muere los dos deben repetir el nivel ya que la idea es que los 2 pases juntos todos los niveles para poder ganar el juego.
	  \Articulo
	

	\ end { enumerate }
	

	

	\ end { documento }

