\documentclass{article}
\usepackage[utf8]{inputenc}
\usepackage[spanish]{babel}
\usepackage{listings}
\usepackage{graphicx}
\graphicspath{ {images/} }
\usepackage{cite}

\begin{document}

\begin{titlepage}
    \begin{center}
        \vspace*{1cm}
            
        \Huge
        \textbf{Proyecto final: Los primeros pasos}
            
        \vspace{0.5cm}
        \LARGE
        Ideación
            
        \vspace{1.5cm}
            
        \textbf{Juan Pablo Cruz Gómez\\
                Erika Dayana León Quiorga}
            
        \vfill
            
        \vspace{0.8cm}
            
        \Large
        Despartamento de Ingeniería Electrónica y Telecomunicaciones\\
        Universidad de Antioquia\\
        Medellín\\
        Septiembre de 2021
            
    \end{center}
\end{titlepage}

\tableofcontents
\newpage
\section{DESCRIPCIÓN DEL JUEGO} \label{contenido}
\begin{enumerate}
  \item \textbf{IDEA GENERAL:} Es un juego en el que el jugador tendrá que moverse por la pantalla evitando los diferentes obstáculos que se encuentran en ella. El objetivo será llegar a un a un lugar determinado en la pantalla en donde se pasará al siguiente nivel. Cada jugador iniciará con un número determinado de vidas para luego ir recolectando más en cada nivel, si las vidas se agotan el juego se reiniciará en el primer nivel.
  \item \textbf{AVATAR:} El personaje principal de nuestro juego irá avanzando a través del mapa sorteando los diferentes obstáculos y proyectiles presentados, para sortearlos tendrá la posibilidad de saltar los obstáculos y esquivar o contraatacar los proyectiles lanzando otros. En cuando a su aspecto físico trataremos de mantenerlo simple sin dejar atrás la estética del juego.
  \item \textbf{ENTORNO:} Nuestro objetivo como desarrolladores del juego será que este sea lo más funcional y fluido posible, es por esto que la estética visual se mantendrá simple pero atractiva para el jugador. A continuación se mencionarán algunas características del entorno:
    \begin{itemize}
        \item Plataformas: Será por donde el avatar irá avanzando para llegar a la meta, se presentan en varios niveles horizontales a los cuales el avatar accederá saltando a través de la pantalla. En cada una de estas se encontrarán obstáculos, proyectiles y herramientas.
        \item Obstáculos:
        \newline
        -	Plataforma rota: Este obstáculo será presentado como explosiones que romperán la plataforma por la cual se desplaza el avatar, las cuales irán explotando al momento de iniciar el juego, es decir, visualmente al iniciar se verán todas las plataformas completas, pero trascurrirán una serie de explosiones que darán lugar a un espacio en la plataforma que el avatar debe saltar.
        \newline
        -   Exámenes materia bajo rendimiento: Uno de los obstáculos que deberá ser evitado por el jugador será el de los exámenes, estarán repartidos en lugares determinados de la pantalla y si son colisionados por el avatar resultará en la pérdida de una de sus vidas.
        \item Proyectiles: Estos serán representaciones de hojas de quizes y serán lanzados por el antagonista a lo largo de cada una de las plataformas. Deberán ser esquivados por el avatar o ser contrarrestados con “Contraataques de conocimiento”.
        \item Herramientas: Estas serán representaciones de libros que se irán recogiendo a lo largo de las plataformas para ser usadas como contraataque a los proyectiles. También se repartirán por la pantalla algún tipo de recompensas que el jugador irá recolectando para sumar puntuación, la que ayudará a la obtención de más vidas.
    \end{itemize}
  \item \textbf{ANTAGONISTA:} El antagonista será una representación de un profesor que lanzará los proyectiles en algunas de las plataformas y se irá moviendo de forma vertical por la pantalla en el costado izquierdo de esta. El antagonista no tendrá variaciones en el transcurso del juego.
  \item \textbf{DIFICULTAD:} La dificultad aumentará a medida que avance el juego, esto se verá reflejado en la cantidad de obstácúlos en pantalla y la velocidad con la que los proyectiles son lanzados.
\end{enumerate}


\end{document}
